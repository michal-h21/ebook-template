% !TEX program = xelatex
% !BIB program = biblatex
\documentclass[12pt]{article}
% -------------------------------------- The below preambles are only necessary when you try to compile the article invididually.

\usepackage{../articlestyle}

% -------------------------------------- The above preambles are only necessary when you try to compile the article invididually.

\begin{document}

% \begin{figure}[t]
%         \centering
%         \includegraphics[width=\textwidth]{article1/image.png}
% \end{figure}

\title{১ম নিবন্ধের শিরোনাম}
\author{\href{https://github.com/rafisics/ebook-template}{১ম নিবন্ধের লেখক}}
\date{}

% \maketitle                                % Use \maketitle if you want to compile it individually

\section{সেকশন}

\begin{figure}[hbt!]
        \centering
        \includegraphics[width=\textwidth]{article1/image.png}
        \caption{Donec vehicula augue eu neque.}
\end{figure}

\firstword{Y}{ou} can put a drop-cap on the first word.
\blindtext

% https://tex.stackexchange.com/a/119985/114006
\begin{figure}[hbt!]
    \centering
    \begin{subfigure}[t]{0.5\textwidth}
        \centering
        \includegraphics[height=1.2in]{example-image-a}
        \caption{Image of A}
    \end{subfigure}%
    % ~
    \begin{subfigure}[t]{0.5\textwidth}
        \centering
        \includegraphics[height=1.2in]{example-image-b}
        \caption{Image of B}
    \end{subfigure}
    \caption{Comparison between the images of A and B}
\end{figure}

% https://www.overleaf.com/learn/latex/How_to_Write_a_Thesis_in_LaTeX_(Part_3):_Figures,_Subfigures_and_Tables

\begin{figure}[hbt!]
     \centering
     \begin{subfigure}[t]{0.3\textwidth}
         \centering
         \includegraphics[width=\textwidth]{example-image}
         \cprotect\caption{\verb|example-image|}
         \label{fig:example}
     \end{subfigure}
     \hfill
     \begin{subfigure}[t]{0.3\textwidth}
         \centering
         \includegraphics[width=\textwidth]{example-image-a}
         \cprotect\caption{\verb|example-image-a|}
         \label{fig:a}
     \end{subfigure}
     \hfill
     \begin{subfigure}[t]{0.3\textwidth}
         \centering
         \includegraphics[width=\textwidth]{example-image-b}
         \cprotect\caption{\verb|example-image-b|}
         \label{fig:b}
     \end{subfigure}
        \caption{Three example images}
        \label{fig:three graphs}
\end{figure}

\lipsum[3]
% https://wwlw.overleaf.com/learn/latex/Wrapping_text_around_figures

\begin{wrapfigure}{r}{0.4\textwidth}                 % https://tex.stackexchange.com/a/56177/114006
        \centering
        \rule{4cm}{6cm}
        \cprotect\caption{Wraped figure using \verb|\begin{wrapfigure}|}           % https://tex.stackexchange.com/a/8814/114006
\end{wrapfigure}

\blindtext

\begin{wrapfigure}{R}{0.5\textwidth}                 
        \centering
        \includegraphics[height=1.2in]{example-image-golden}
        \caption{Image of Golden Ratio}
\end{wrapfigure}

\lipsum[1-4]

% % generates a paragraph of dummy lorem ipsum text
% \blindtext

% generates multiple paragraphs of dummy lorem ipsum text
% \Blindtext

% % generates whole document with dummy lorem ipsum text
% \Blinddocument

\end{document}
