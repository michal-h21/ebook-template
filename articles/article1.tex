% !TEX program = xelatex
% !BIB program = biblatex
\documentclass[12pt]{article}
% -------------------------------------- The below preambles are only necessary when you try to compile the article invididually.

\usepackage{../articlestyle}

% -------------------------------------- The above preambles are only necessary when you try to compile the article invididually.

\begin{document}

% \begin{figure}[t]
%         \centering
%         \includegraphics[width=\textwidth]{article1/image.png}
% \end{figure}

\title{১ম নিবন্ধের শিরোনাম}
\author{\href{https://github.com/rafisics/ebook-template}{১ম নিবন্ধের লেখক}}
\date{}

% \maketitle                                % Use \maketitle if you want to compile it individually

\section{সেকশন}

\begin{figure}[hbt!]
        \centering
        \includegraphics[width=\textwidth]{article1/image.png}
        \caption{Donec vehicula augue eu neque.}
\end{figure}

\firstword{Y}{ou} can put a drop-cap on the first word.
\lipsum

\begin{figure}[hbt!]
    \centering
    \begin{subfigure}[t]{0.5\textwidth}
        \centering
        \includegraphics[height=1.2in]{example-image-a}
        \caption{Image of A}
    \end{subfigure}%
    ~
    \begin{subfigure}[t]{0.5\textwidth}
        \centering
        \includegraphics[height=1.2in]{example-image-b}
        \caption{Image of B}
    \end{subfigure}
    \caption{Comparison between the images of A and B}
\end{figure}

% % generates a paragraph of dummy lorem ipsum text
%\blindtext

% generates multiple paragraphs of dummy lorem ipsum text
% \Blindtext

% % generates whole document with dummy lorem ipsum text
% \Blinddocument

\end{document}
