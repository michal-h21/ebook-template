% !TEX program = xelatex
% !BIB program = biblatex
% !TEX root=../main.tex
\documentclass[12pt]{article}
% -------------------------------------- The below preambles are only necessary when you try to compile the article invididually.

\usepackage{../articlestyle}

% -------------------------------------- The above preambles are only necessary when you try to compile the article invididually.

\begin{document}

% \begin{figure}[t]
%         \centering
%         \includegraphics[width=\textwidth]{example-image}
% \end{figure}

\title{২য় নিবন্ধের শিরোনাম}
\author{\href{https://github.com/rafisics/ebook-template}{২য় নিবন্ধের লেখক}}
\date{}

% \maketitle                                % Use \maketitle if you want to compile it individually

\begin{figure}[htbp]
        \centering
        \includegraphics[width=0.8\textwidth]{example-image}
        \caption{Donec vehicula augue eu neque.}
\end{figure}

\firstword{Y}{ou} can put a drop-cap on the first word. \blindtext \\

\section*{সেকশন: List Styles}

Using \verb|\begin{itemize} ... \end{itemize}|
\begin{itemize}
        \item Item one
        \item Item two
\end{itemize}

Using \verb|\begin{myitemize} ... \end{myitemize}|
\begin{myitemize}
        \item Item one
        \item Item two
\end{myitemize}

Using \verb|\begin{mylist} ... \end{mylist}|
\begin{mylist}
        \item Item one
        \item Item two
\end{mylist}

Using \verb|\begin{enumerate} ... \end{enumerate}|
\begin{enumerate}
        \item Item one
        \item Item two
\end{enumerate}    

\lipsum[3][1-6]

\section*{সেকশন: Quote Styles}

Example 1. \verb|\chapquote{Quote}[Author][source]|
\chapquote{Lorem ipsum dolor sit amet, consectetur adipiscing elit.}[Quote Author][Quote Source]

Example 2. \verb|\chapquote{Quote}[Author][]|
\chapquote{Lorem ipsum dolor sit amet, consectetur adipiscing elit.}[Quote Author][]    

Example 3. \verb|\chapquote{Quote}[][source]|
\chapquote{Lorem ipsum dolor sit amet, consectetur adipiscing elit.}[][Quote Source]    

Example 4. \verb|\chapquote{Quote}| 
\chapquote{Lorem ipsum dolor sit amet, consectetur adipiscing elit.}

Example 5. \verb|\quote{Quote}| 
\quote{Lorem ipsum dolor sit amet, consectetur adipiscing elit.}

% \lipsum[3][1-6]

% % generates a paragraph of dummy lorem ipsum text
% \blindtext

% generates multiple paragraphs of dummy lorem ipsum text
% \Blindtext

% % generates whole document with dummy lorem ipsum text
% \Blinddocument

\end{document}
